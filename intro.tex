\documentclass[autodetect-engine,dvi=dvipdfmx,ja=standard,a4j,12pt]{bxjsarticle}

\usepackage{amsmath}
\usepackage{ascmac}
\usepackage{caption}
\usepackage{fancybox}
\usepackage{hyperref}
\usepackage[dvipsnames]{xcolor}
\usepackage{listings, jvlisting}
\usepackage{forest}
\usepackage{tcolorbox}

\title{\textbf{プログラム開発への取り組みかた} \\
        | \Large{初学者向け} |}

\author{作成:寺岡 久騎}
\date{最終更新日:\today}

\begin{document}
\maketitle

\begin{abstract}
この資料は,プログラミングに初めて挑戦する,あるいは
プログラミングによる開発作業に慣れていない部員向けに作成したものです.
ある程度"できる人"からすれば間違えていることを書いてしまっているかもしれません.
予めご了承下さい.また,資料の利用については部内限定でお願いします.
\end{abstract}

\tableofcontents

\clearpage
\part*{まえがき}
こんにちは,資料作成者の寺岡です.所属は
工学部の情電数理DS系情報工学コース,学年は3回生です.ロボ研
では制御班として活動しています.この資料に書いてあることはあくまで
僕自身がプログラミングやそれに関わる開発を行ったり,人に教えたりした時の経験から
大事だと感じたことを基にしたものであり,当然全ての人にとって役立つとは限らないです.
なので,もし資料を読んでいて「自分はそうは思わない」「こんなことより別の方法が良い」
などと感じたら,もちろんそのご自身の感性に従って下さい.向き不向きや趣向は個人差があるので
この資料に書かれていることを鵜呑みにする必要はありません.

\clearpage

\section{プログラミングはどう勉強するべき?}
ロボット製作において必須となる要素の一つである「プログラミング」,
制御班では避けては通ることのできない超重要な作業です.
もちろん制御班としてではなく,特に工学部では講義の中や更には個人でも興味があって触れることが多いでしょう.
ですが,いざ勉強しようとしても
\begin{center}
    \textbf{「どのプログラミング言語を勉強すればいいのか?」\\
    「どこから着手するべきなのか?」\\
    「何を使って勉強するべきなのか?」}
\end{center}
というように始めるまえから色々な疑問が湧くことがあると思います.さらに言えば,
実際に勉強し始めても
\begin{center}
    \textbf{「何を目標にすればいいか分からない」\\
    「何を学習材料にするのが正解か分からない」\\
    「概念が抽象的すぎて全然理解できない」} 
\end{center}
といった様につまづいてしまうことも多々あると思います.
これらは実際に僕自身が体験しましたし,他の人を見てもこのようなことを感じたり経験している
ことが多いようです.

この様なことが起こる原因として,僕は
\begin{center}
    \textbf{\Large{プログラミングに対するイメージ}}
\end{center}
が大きいと感じています.上記に挙げた「概念が抽象的」
であることで理解が難しく感じてしまうことなども,このプログラミングに対する
先入観やイメージによるものである可能性が高いです.
\clearpage

ではこれからは,つまづきやすい事柄が多いプログラミングの勉強について,
実際にどう取り組むべきかやどういうイメージをもつのが良いかを
以下の3段階に分けて紹介します. 
\begin{itemize}
    \item 勉強を始める前の段階
    \item 勉強をし始めた段階
    \item 少し慣れた段階
\end{itemize}

\section{勉強を始める前の段階}

\subsection{プログラミングに対する意識・イメージ}
プログラミングの勉強に関する資料なのに始める前の事がそんなに重要なのか?と感じてしまうかもしれません.
ですが,僕は正直この段階が一番重要なんじゃないかと思っています.
これは先述したように,プログラミングの勉強におけるつまづきの
大きな原因に「プログラミングに対するイメージ」が関わっていると感じるからです.
読者の方はどのようなイメージを持っていますでしょうか?
\begin{center}
    \textbf{
        「なんか難しそう」\\
        「複雑で理解できなさそう」\\
    }
\end{center}
こういうイメージを持つ方が多いかもしれません.これが最初にして最大の壁です.
\begin{center}
    \textbf{
        難しそうだと考えてしまっているがために,変に深く理解しようとしたり,
        手が付けにくくなってしまうことで後の学習が大変に感じてしまうのです.
    }
\end{center}

恐らくこれは「プログラミング言語」が複雑で抽象的なイメージであるコンピュータ
で使われることが原因なのでしょう.ですが,始めの段階から勉強を進めていく上でこの印象を
抱き続けるのは間違いです.
実際に取り組む上で重要なのは,
\begin{center}
    \Ovalbox{\Large{\textbf{プログラミング言語は「言語である」という事です}}}  
\end{center}



\subsection*{プログラミング言語は「言語である」}
今まで「英語」は学校の授業や個人学習でどのような方法で勉強したでしょうか.
多少の差異はあれども,
\begin{itemize}
    \item 単語とその意味を覚える
    \item 基本的な文法を覚えて適切に使い分ける
\end{itemize}
ということはしてきたはずです.ですが,
「なぜbe動詞にはam, are, isが使われるのか」
「なぜ過去分詞形がそのような形式なのか」
といった歴史や背景,意図には触れられることはあまり無いと思います.
恐らくそれは「言語を使う上ではさほど重要では無いから」でしょう.
日本語でもそうですが,人とコミュニケーションを取る手段として言語を使う上で,
その言語の成り立ちや根本的な構造・歴史的背景を全て理解して使用するなんて普通しないですよね.
それは,言語はあくまでコミュニケーションの「手段」であるからです.
\begin{center}
    \underline{\Large{プログラミング言語もこれと同じです}}
\end{center}

コンピュータが機械語という0と1により構成された命令しか
読み取れず,人間がプログラムを書くことが非常に困難であるがために,
人間が理解しやすい形でプログラミングできるように開発されたのが「プログラミング言語」です.
つまり,プログラミング言語はコンピュータと人間の橋渡しを行う「言語」であり「手段・道具」なのです.
ですから,プログラミング言語を初めて学ぶ際に「なぜこう書かないとプログラムが動かないのか」「なぜこの構文を書けばこんな処理が行われるのだろう」
などの様に内部構造やバックグラウンドを考えるのは\textbf{間違い}です.
もちろん言語をある程度使えるようになってから,さらなる上達を目的にしていたり,プログラムを効率良く動作させたり安全性を考慮した設計をするうえでは
内部の仕組み・バックグラウンドでの動作を知ることは必要ですが,まだ慣れていないプログラミング言語を
ひとまず使えるようになるための勉強においてはほとんど不要です.
\begin{center}
    \textbf{プログラミングにおける最初の学習のステップとして大切なのは,\\まず「使えるようになること」です.}
\end{center}

つまり僕が伝えたいのは,プログラミングに対しては
\begin{center}
    「複雑なものである」\\
    「難しい内容を論理的に考えて理解しなければならない」
\end{center}
ではなく,
\begin{center}
    「言語として使えるようになれば良い」\\
    「とりあえずコンピュータ上で動けばOK」\\
    「こうすればこう動くのかー」
\end{center}
ぐらいのイメージや気持ちで挑むことが大切ということです.そうすれば
ハードルも低く感じられて,勉強のやる気が一気に削がれることもないでしょう.

\begin{center}
    \Large{\textbf{まとめ}}
\end{center}
\begin{itemize}
    \item プログラミング言語は人間とコンピュータとのコミュニケーションに使う「言語」である
    \item プログラミング言語もまず「理解する」より「使えること」が重要
    \item プログラミングの学習において,最初は「とりあえず言語を使って動くプログラムが書けるようになる」ことを目標にする
    \item 「こう書けばこう動くのかー」ぐらいの気持ちで挑むと良し
\end{itemize}

\subsection{学習材料には何をどう使うべき?}
さて,前節ではなるべくつまづいたり挫折しないために,
プログラミングに対するイメージについて僕自身大切だと思うことを書きました.
ここからは実際に手を動かして勉強を進めていく時に迷いがちな「学習材料」
についての話をしようかと思います.これもプログラミングの勉強の効率を
左右する話題の一つです.

プログラミングの学習材料としては主に
\begin{itemize}
    \item 書籍
    \item ネットの記事・動画
\end{itemize}
の二つが挙げられます.どちらか一方のみを使用するというのは稀なケースだと
思うので,この2つについて,それぞれのメリットと
どのようなケースで使うのがおすすめかを記します.\\

\begin{itembox}[c]{\huge \ \textbf{書籍}\ }
\begin{center}
    {\large \textbf{メリット①:体系的に学べる}}
\end{center}

書籍の最大のメリットはこれでしょう.特に入門書などの基礎を学ぶ
ための内容であれば,
\begin{itemize}
    \item 重要な事柄を丁寧に書いている
    \item 例やサンプルがたくさん記載されている
\end{itemize}
といった様に分かり易い形での記述が多く,1冊読み切ればその言語を一通り使えるように
なるくらいの知識をまんべんなく学べると思います.
実際,読み進めつつサンプルプログラムを書きながら勉強すると
結構頭に入るのでおすすめです.

\begin{center}
{\large \textbf{メリット②:復習しやすい}}
\end{center}
これは①にも関わりますが,基礎的なことがまとまっていることで
少し忘れてしまった内容をすぐに探して見直せるのは大きいメリットだと思います.

\begin{center}
    {\large \textbf{デメリット①:お金がかかる}}
\end{center}
これは仕方ないことですが,やはり内容がしっかりした物になれば
それなりに値が張ります.高いものだと1万円弱のものもあったりするので
買うときはきちんと選ぶことが大切です.

\begin{center}
    {\large \textbf{デメリット②:書店に数が無い}}
\end{center}
これが結構悩ましいポイントです.プログラミングの書籍は
一般的には少しマニアックなジャンルであるせいか,大きめの店舗でも全然バリエーションが
無かったり,そもそも置いてある数が少ないことがあるので残念です.
ネットで購入したり電子版を利用するのも良いと思いますが,買う前に内容とその雰囲気を
吟味できないというのが個人的にデメリットだと感じてしまいます.
\end{itembox}
\clearpage

個人的におススメの書店や書籍のジャンルを記載しておきます.参考程度にどうぞ.
\begin{center}
    {\Large \textbf{書店(図書館)}}
\end{center}
\begin{boxnote}
{\large \textbf{・岡山大学付属図書館}} \\
やはり専門書とくれば大学図書館が結構強いですね.
入ってすぐ奥の別館(?)1階の机が並んでいるスペースのすぐ横の棚一面に
プログラミング関連の書籍が結構並んでいます.割と古めの本が多く,
そこそこマニアックな内容のものもあるので,どちらかというと入門というより
中級者向けな感じがします.無料で利用できるので是非積極的に足を運んでみると
いいでしょう.通常であれば貸出期間が2週間くらいですが,長期休暇であれば2か月間借りられるのがGood.\\

{\large \textbf{・マスカットユニオン BookStore}} \\
やはり大学施設は専門書のバリエーションが豊富です.入門書であれば
ここを探せば大体良いのが見つかります.図書館とは違ってマニアックな内容のものは少ないかもしれません.\\

{\large \textbf{・丸善 \ 岡山シンフォニービル店}} \\
大学施設以外では一番おススメです.岡山駅周辺で訪れた書店の中では
プログラミング関連書籍のバリエーションと在庫数は圧倒的No.1で,
マスカットユニオンBookStoreと同等かそれ以上だと思います.
入門書から上級者向けのものまで幅広く置かれているのが非常に良いです.
プログラミングだけではなく,電気・回路系やマイコン,制御工学や
設計学などの工学関連の書籍も豊富ですので,ロボ研部員には一度訪れてもらいたいです.

\end{boxnote}
当然紙媒体だけでなく電子版もおススメです.なんならそっちの方が
パソコンを操作しながら一緒に見れたり,プログラムをコピペできるので良いかもしれません.
まあ紙か電子は完全にお好みです.ですが,やはり体系的に学べるというのは学習するうえで非常に
重要なので是非一度紙媒体の物を手にとって吟味してから自分に合ったものを選んでみてほしいです.
上記に示した所以外でも品揃えが良いところはあるでしょうから,
是非書店等を巡ってみてはいかがでしょうか.

\begin{center}
    {\Large \textbf{初学者におすすめな書籍の特徴}} \\
\end{center}
初めて学習する場合,どのような書籍を選ぶべきか割と悩んでしまうと思います.
ですので,個人的にこんな特徴の本を選んでおけば,分かりやすく学べてあんまり後悔しないんじゃないかな
と思うものをいくつか紹介します. 

\begin{boxnote}
{\large \textbf{・「やさしい〇〇」「基礎から学ぶ〇〇」といったタイトルの本}} \\
当然といえばそうですが,初めて手を付けるプログラミング言語であれば明らかに初心者向け
のタイトルのものであれば大丈夫だと思います.ほとんどの場合,環境構築の手順とかも載っているのでかなり良心的です.
意外にも「入門」と銘打ったタイトルの癖にその言語の基礎を知っている前提の中級者向けの本があったりするので注意が必要です.

{\large \textbf{・図やイラスト,サンプルがたくさん載っている本}} \\
これが結構重要だと思います.プログラミングは抽象的な概念のオンパレードなので,
初めて学習するときに文章と断片的なソースコードだけでは不十分だと感じます.
なので,図やイラストが豊富で読んでいてイメージしやすい形式のものを選ぶと良いでしょう.
\end{boxnote}

\begin{itemize}
    \item 書籍はがっつり学習したいときに使う
    \item 内容は自分に合ったもの・読みやすいものを選ぶ
\end{itemize}
このくらいのポイントを把握しとけばさほど問題はないでしょう.
何度もしつこいかもしれませんが,「体系的に学べて,かつ内容がイメージしやすい」ものを
吟味することがおススメです!

\clearpage

\begin{itembox}[c]{\huge \ \textbf{ネット上の記事・動画}\ }
\begin{center}
    {\large \textbf{メリット①:お金と時間がかからない}} 
\end{center}
書籍には無い最大のメリットです.無料なのもそうですが,PCや
タブレットなどで学習するのは非常に効率が良く,勉強するときの負担も少ないのが素晴らしいです.
特にプログラミングの入門や基礎に関してはネットに山ほど情報が転がっているので,
学習し始めの頃は結構お世話になります.

\begin{center}
    {\large \textbf{メリット②:追加の知識・情報が得やすい}} 
\end{center}
ネットの記事は基本的にその内容に精通している人が書いている場合がほとんどです.
故に基本的な内容に加えて,あまり本には載っていない知識や少々マニアックな情報などが
得られる可能性が高いです.

\begin{center}
    {\large \textbf{デメリット①:情報が断片的である}} 
\end{center}
これはネット記事やWebそのものの特性が関係していますが,
やはり膨大な内容を丁寧に記述しているものはあまり多くないと感じます.
記事を分けて体系的に学べるように工夫されている方も多いですが,
書籍と比較して図やイラスト・ソースコードの量が少ないのが残念なポイントです.
\end{itembox}

ネット記事・動画は初学者の学習のおいてはがっつり学ぶときに使うというよりは
気になった箇所の復習みたいに補助的に使うイメージがあります.ですが,
YouTubeには長時間でプログラミングの基礎を教えてる動画があったりするので,
人によっては本よりもこっちの方が勉強しやすいということもあるでしょう.

また,環境構築でつまづいたり,実際にプログラムを動かしていてエラーが発生したり
した時には,調べれば対処法が得られるのが結構重要です.このように問題に直面した際はほとんどの場合
ネットで調べるため,利用しないという日はありません.自分が得たい情報を調べる能力というのも
プログラム開発においては重要なので,どんどん利用しましょう.

\clearpage

\begin{center}
    {\Large \textbf{よく利用するWebサイト}} \\
\end{center}
プログラム開発において結構お世話になるであろう有益なWebサイト
をいくつか紹介します.

\begin{boxnote}
{\large \textbf{・Qiita}} \\
Qiitaはエンジニアに関する知識を記録・共有することを目的としたWebサービスで,
有識者が多く利用していることもあり,プログラム開発で気になったことを調べれば大体このサイトの記事が引っ掛かります.
正直毎日お世話になっている.

{\large \textbf{・Zenn}} \\
ZennはQiitaとほぼ同じで,エンジニアの情報共有サービスです.
収益化に関わる部分が若干違うらしい.QiitaとZennがネット上での
プログラム開発の調べものをするときの2台巨頭というイメージ.

{\large \textbf{・Stack OverFlow}} \\
これも上2つと同様の情報共有コミュニティ.
プログラムを実行したときのエラー内容を調べたりするとよく記事が引っ掛かる.
海外の利用者も多く,結構マニアックなことを調べてもヒットすることがある.
\end{boxnote}

\begin{center}
    {\Large \textbf{まとめ}}
\end{center}
\begin{itemize}
    \item 書籍は言語を体系的にがっつり学ぶのにおススメ
    \item 始めは図やイラストなどのイメージがしやすい内容のものを選ぶと良いかも
    \item ネット記事や動画はエラー対処や調べものの様に補助的に使うイメージ
    \item ネット記事・動画でも全然基礎的なことを学べる
\end{itemize}
個人的には書籍をメインで勉強しつつ,気になった箇所をネットで調べるという
スタイルがおすすめ.書籍は自分に合った内容を選び,ネットの記事は
複数調べて多角的に情報を集めると,より理解が深まると思います.

\subsection{どのプログラミング言語を勉強しよう?}
プログラミング学習を始めるときにはこれも迷うのではないでしょうか.
学習する言語の選び方などには以下の様な意見が挙げられると思います.

\begin{enumerate}
    \item 興味のある分野(または学部・学科の専門分野)に使われている言語を探す
    \item 汎用性の高い言語を選ぶ
    \item 流行りの言語を触る
\end{enumerate}
これらの内どのような基準で選ぶとよいのかを僕なりの意見を交えて紹介します.

\subsubsection{興味のある分野・自分の専攻に関係のある言語}
\paragraph*{興味のある分野に関連した言語}
個人的に言語を選ぶ際の基準としてはこれを重視するべきだと考えています.
というのもこれは「学習のモチベーションに直結する」と思うからです.
プログラミングはどうしても抽象的な概念が多く,ただ目的もなく勉強していると
どうしても飽きがきたり,理解するのが苦しく感じたりすることがあります.
ですが,何か作りたいものがあったり言語が得意する分野に興味があってモチベーションがある状態
ならばその過程もあまり苦に感じず学習が捗り,継続的な学習に繋がるでしょう.

\paragraph*{専攻に関係がある言語}
僕個人としてはこれを基準にするのも悪くないと思っています.
というもの,プログラミング言語に関わらず自身の専攻分野で使用する知識は
大学での講義内容や研究,さらには就職した先でも頻繁にする使用するからです.
当然ですが,その様な事柄を大学の講義で習うだけでなく自分から
進んで学んで習得・理解すれば大きなアドバンテージになります.
実際,僕はロボ研の活動の中で専攻である情報工学と密接な関係があるC言語を自ら学習していたことで
1年後期のプログラミングの授業や2年生の専門科目の実践的な演習などが
理解しやすくなりました.

専攻分野自体でなくとも,プログラミングをする講義で使用する言語
などを触っておけば結構余裕ができるのでおススメです.

\subsubsection{汎用性の高い言語}

\subsubsection{流行りの言語}

\section{勉強し始めの段階}


\section{少し慣れた段階}

\end{document}